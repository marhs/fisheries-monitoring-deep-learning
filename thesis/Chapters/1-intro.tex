% Chapter 1

\chapter{Introducción} % Main chapter title
\label{Chapter1} % For referencing the chapter elsewhere, use \ref{Chapter1} 

%----------------------------------------------------------------------------------------

% Define some commands to keep the formatting separated from the content 
\newcommand{\keyword}[1]{\textbf{#1}}
\newcommand{\tabhead}[1]{\textbf{#1}}
\newcommand{\code}[1]{\texttt{#1}}
\newcommand{\file}[1]{\texttt{\bfseries#1}}
\newcommand{\option}[1]{\texttt{\itshape#1}}
%----------------------------------------------------------------------------------------


Una de las ventajas de las nuevas tecnologías es la cantidad de datos que son
capaces de generar para problemas específicos. Donde antes existía una
necesidad de una persona recolectando datos de forma manual ahora existen
maneras de automatizar ese proceso. Este trabajo es uno de esos procesos.

La competición propuesta a través de la plataforma \textit{Kaggle} que se
quiere resolver aquí tiene como objetivo clasificar peces dentro de imágenes de
barcos pesqueros. Esto permitirá en un futuro predecir patrones migratorios,
estudiar patrones de pesqua o vigilar que la pesca sea una actividad que no
dañe al medio ambiente. Sin embargo es necesario encontrar un sistema que
permita distinguir diferentes peces tal y como lo haría un humano.

El desarrollo de redes neuronales convolucionales en arquitecturas de
aprendizaje profundo está consiguiendo precisiones similares a la humana en
problemas de reconocimiento visual \parencite{taigman}. Esto hace que sea cada
vez más interesante aplicar este tipo de soluciones a problemas de
clasificación visual donde antes no era posible.

En este trabajo se intentarán aplicar técnicas de aprendizaje profundo para
resolver este problema, variando arquitecturas y usando redes preentrenadas. 

