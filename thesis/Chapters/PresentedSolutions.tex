% Chapter Template

\chapter{Soluciones presentadas} % Main chapter title

\label{ChapterX} % Change X to a consecutive number; for referencing this chapter elsewhere, use \ref{ChapterX}

\section{Idea}
Cuando un humano afronta el problema de clasificar las imágenes se puede
observar que se sigue una estrategia clara: primero buscamos si
existe un pez en la foto y luego intentams clasificarlo en una de las categorías
disponibles. 

El acto de encontrar un pez en la foto puede variar de una persona a otra,
pero se podría generalizar diciendo que es necesario encontrar una serie de
características que puedan ser identificadas con alguna de las categorías con
las que se trabaja.

La idea de la solución parte de esta base. A la hora de clasificar una
imagen primero es necesario encontrar el contenido relevante para, luego, 
usarlo en la clasificación. En el presente problema esto se puede traducir a que
primero debemos encontrar los peces, si existe alguno en la imagen, antes de
decidir a qué categoría pertenecen.

figura: diagrama de decision sobre el proceso mental:
encontrar caracteristicas?: no: NoF, si: encaja con pez?: no: other, si: pez

Esta idea nos deja una estructura clara en mente. Una parte del modelo debe
encontrar las características necesarias para clasificar y otra debe clasificar
en base a estas características. Las arquitecturas encontradas en problemas
similares (citas al paper de vgg y el de resnet), permitirían separar con 
claridad estas dos etapas, pudiendo experimentar con diferentes modelos en ambos
casos.

\section{Arquitectura}
% TODO no hay indentacion en este parrafo
La estructura de la solución general sigue el esquema de la figura X. Lo que se
detalla aquí es la arquiectura general de las soluciones, sobre la cual se ha
ido iterando probando diferentes elementos y valores.

Consta de dos partes, un modelo general preentrenado y un modelo
consecutivo al primero entrenado de cero para este el problema. La idea
de esta arquitectura es aprovechar la potencia de un entrenamiento genérico de
reconocimiento de imágenes y adaptarlo a nuestro problema mediante una capa
separada e intercambiable.

Las ventajas que ofrece esta arquitectura son claras: es posible separar el
problema en un problema de reconocimiento de imágenes y otro problema de
clasificación.

\subsection{Modelo preentrenado}

Aqui detallo las implementaciones de los modelos basados en Imagenet (los
disponibles en keras.io): VGG, ResNet e Inception. Tambien tengo que decir
porque prefiero VGG a otros modelos, sobre todo a nivel didactico.

\subsection{Fine tuning}
